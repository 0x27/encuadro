\documentstyle[twoside,WICSBOOK]{article}
\setlength{\unitlength}{1mm}

\title{\vspace{-3pc}\titlesize\bf Low-level Grouping of Straight Line Segments}

\author{\large A. Etemadi\\ 
\large J-P. Schmidt, G. Matas, J. Illingworth, and J. Kittler\\ 
\normalsize Department of Electronic and Electrical Engineering, University of Surrey\\ \normalsize Guildford, United Kingdom \date{}}
\begin{document}
\maketitle

\begin{abstract}\ninesize 
\noindent  In this paper we present a formalism for the formation of self 
consistent, hierarchical, ``Low-Level'' groupings of pairs of straight line 
segments from which all higher level groupings may be derived. Additionally, 
each low-level grouping is associated with a ``Quality'' factor, based on 
evidential reasoning, which reflects how much the groupings differ from 
mathematically perfect ones. This formalism has been incorporated into 
algorithms within the ``LPEG'' software package produced at the University of 
Surrey. LPEG was developed as part of the Vision As Process \cite{crowley89} 
project. We present results of the application of these algorithms to sets of 
line segments extracted from a test image. 

\end{abstract}

\section{Introduction}

 Grouping of straight line segments has been the subject of much 
investigation. Most research in this field has been concentrated on forming 
perceptually significant groupings. The reader is 
therefore referred to commonly available bibliographic databases on the 
subject with special reference to \cite{weiss86}, \cite{lowe87}, 
\cite{mohan89}, \cite{horaud90}, and \cite{faugeras90} and references 
therein. Many types of groupings of two or more line segments have been 
proposed in the literature. In general the proposed groupings either fail to 
allow the consistent formation of higher level groupings or involve 
heuristics. 

 Let us first consider the possible relationship between any pair of 
lines. Clearly they may either be collinear, parallel, or intersecting. When we
consider line segments we may further subdivide the parallelism relationship
into overlapping, and non-overlapping. The intersecting pairs may also be 
divided according to whether the intersection point lies on either, both, or
only one of the line segments. These then are the complete set of 
relationships between two line segments. Clearly all higher level groupings may
be formed by combining subsets of this set. 

 In this paper it is our aim to first define a formalism for the formation of 
this set, and show how this formalism allows us to implement these groupings
in a manner useful for vision. In the first section we discuss what properties 
are imposed on algorithms for the formation of these groupings if 
they are to be useful for vision purposes. In the following sections we discuss
each of the proposed groupings in turn. Associated with each grouping is a
``Quality'' factor based on comparing the observed grouping with an ideal 
case. This factor greatly simplifies the control of image interpretation tasks
and is defined separately for each grouping in turn. We present results 
obtained using the ``LPEG'' software package within which these algorithms 
have been implemented. Finally we summarize our findings, and discuss this 
work in the context of a complete vision system.

\section{Low-Level Groupings}

 The most desirable feature of any proposed grouping is consistency.
Strictly mathematical definitions of parallelism, collinearity, and 
intersection ensure this property. However so long as our definitions are 
consistent we need not adhere to these mathematical definitions. We may choose 
for example to label any pair of line segments forming an acute angle of 10 
degrees as ``parallel''. To ensure consistency we define ``non-parallel'' 
segments as those forming acute angles greater than 10 degrees.

 Self consistency is an essential requirement since it ensures that the 
proposed grouping relations are independent of the order in which the line 
segments are chosen.  For the purposes of computer vision, scale independence 
is another important requirement, since in general the conversion factor 
between distances in pixels measured across an image is not known or is 
poorly defined. Finally we also wish to be able to form a hierarchy of 
groupings which will allow us to filter unlikely combinations at an early 
stage. 

\begin{figure}[htbp]
\vspace*{-10mm}
\special{psfile=Figures/Groupings.ps}
\vspace*{4cm}
\vspace*{-5mm}
\caption{Low-Level groupings within the LPEG system}
\label{fig:groupings}
\end{figure}

 Based on these requirements we have defined the Low-Level groupings shown in
the form of a tree structure in Figure~\ref{fig:groupings}. The set of straight
line segments forms the top level of the tree. In the first level there are 
the three main groupings, namely Parallel, Collinear, and Junction. At the 
next level the Parallel grouping is separated into two further types based on 
whether the line segments overlap. The Junctions are also separated at this 
level into Intersecting and Non-intersecting types, based on whether the 
junction point resides on either, both, or neither of the line segments 
forming the junction. The most important distinctions are between the 
groupings at this level of the tree. The Junction types are finally separated 
to form the lowest level of the tree according to the acute angle between the 
line segments forming the junction. The characters used to denote the junction
types also reflect their form. These junction types are especially useful when 
considering particular applications. For example if we know apriori that the 
image may contain an L shape, such as for character recognition purposes, the 
inclusion of L junctions as part of the hierarchy will allow us to rapidly 
identify this structure. The distinction between intersecting and 
non-intersecting junction types is an important one since it also allows us to 
hypothesize junctions between surfaces. Due to the generally poor segmentation 
the formation of all these groupings requires a statistical model of the 
feature extraction process, or at least a worst-case knowledge of the errors 
involved. 

 The next four sections are devoted to the discussion of the various 
groupings at the lowest levels of the tree. In each case we present, and 
discuss, the definition used in the formation of the grouping in the context of
the above requirements. Associated with each grouping is a ``Quality'' factor 
which enables us to filter unlikely groupings at an early stage. We also show 
how this ``Quality''factor is computed and discuss its relevance in each case.

\subsection{Overlapping Parallel Line Pairs}

 When the acute angle between any pair of straight line segments is below a 
specified value, the line segments are considered as candidates for 
parallelism or collinearity. Note that so long as we are consistent, this 
value is not important. We have chosen a value of 0.06 Radians for this angle
based on digital straight line properties. Having filtered line segments 
which may form parallel or collinear pairs, using the acute angle criterion 
alone, we now show how these pairs may be classified as overlapping parallel.

\begin{figure}[htbp]
\vspace*{-10mm}
\special{psfile=Figures/Parallel.ps}
\vspace*{6cm}
\vspace*{-5mm}
\caption{Overlapping (2a) and non-overlapping (2b) parallel line segments}
\label{fig:parallel}
\end{figure}

 In the following discussion we shall make use of the parameters $L_{i}$, 
$LP_{i}$, $\theta_{i}$, $\sigma^{\parallel}_{i}$, and $\sigma^{\perp}_{i}$, 
which represent the length, the projected length onto the``Virtual Line'', 
orientation angle, and standard deviation of the position of the end points of 
the line segment along and perpendicular to its direction, respectively. 
The subscript {\it i} is used for referencing the line segments. In general the 
standard deviations are used as a means of incorporating the uncertainties in 
the line segment extraction process into the labeling of the groupings at the 
lowest level of the grouping tree. These standard deviations may be replaced 
by constants without affecting the basic grouping algorithms. In order to avoid
ambiguity in the formation of the groupings we have restricted our analysis to
those line segments whose length is greater than the largest sum of any
combination of their $\sigma^{\parallel}_{i}$, and $\sigma^{\perp}_{i}$.

 Given the line segment pair {\it L}1 and {\it L}2 in 
Figure 2a we first attempt to find the ``Virtual Line'' 
{\it VL}. {\it VL} is initially 
defined through its orientation and the point {\bf P} through which it passes. 
The orientation angle of the {\it VL} is given 
by the weighted mean of the orientation of the two line segments as defined by
the equation:
\vspace{-2mm}
\begin{equation}
 \theta_{\it VL} = \frac 
{ L_{1} \times \theta_{1} + L_{2} \times \theta_{2} }
{ L_{1} + L_{2} }
\end{equation}
\vspace{-2mm}
 Note that we have not used the standard deviation of the orientation angles 
of the line segments in the above equation since they are generally a 
function of the line length. Now the x, y position of the point {\bf P} 
through which the Virtual Line passes are similarly defined by:
\vspace{-2mm}
\begin{equation}
 x_{\it VL} = \frac 
{ L_{1} \times x_{1} + L_{2} \times x_{2} }
{ L_{1} + L_{2} }
~~and~~
 y_{\it VL} = \frac 
{ L_{1} \times y_{1} + L_{2} \times y_{2} }
{ L_{1} + L_{2} }
\end{equation}
\vspace{-2mm}
where $x_{i}$, and $y_{i}$ indicate the x, y positions of the midpoints, 
{\bf M}i, of line {\it L}i, respectively. We now compute the positions of the
points P1, P2, P3, and P4 which as defined by the intersection points of
perpendiculars dropped from the end points of the line segments onto the
Virtual Line. The end points of {\it VL} are defined by the pair of
points Pi, and Pj, from the set P1, P2, P3, and P4, separated by the largest 
distance. These points define the length $L^{\it VL}_{1,2}$ of the Virtual 
Line. The line segments {\it L1}, and {\it L2} are defined to be overlapping 
parallel if
\vspace{-2mm}
\begin{equation}
 L^{\it VL}_{1,2} \leq LP_{1} + LP_{2} + 
   \sigma^{\parallel}_{1} + \sigma^{\parallel}_{2}
\end{equation}
\vspace{-2mm}
 The symmetrical nature of the definition of such a grouping ensures self 
consistency. 

 If the sum of the lengths of the two line segments is equal to twice the 
length of the Virtual Line, and the orientations of the line segments are 
equal these segments form a perfect overlapping parallel pair in a strictly 
mathematical sense. For vision purposes we wish to determine how closely the
observed pair deviates from this perfect pairing. Essentially we are 
attempting to determine how much evidence is available for such a perfect
pairing by comparing measured quantities with the ideal case. We now define a 
``Quality'' factor for overlapping parallel line segments which allows such a 
determination.
\vspace{-2mm}
\begin{equation}
 Quality_{OVP} = \frac {LP_{1} + LP_{2}
         - \sigma^{\parallel}_{1} - \sigma^{\parallel}_{2} }
{2.0 \times L^{\it VL}_{1,2}}
\label{equ:qualovp}
\end{equation}
\vspace{-2mm}
 Our definition also ensures that this dimensionless parameter is always in 
the range zero to one. A value of 1.0 for the Quality factor implies a perfect
grouping. This form of the definition implies that the Quality factor for 
parallel overlapping line segments will generally be $\geq$ 0.5. The above 
definition is again self consistent and has the desirable property that it 
degrades monotonically as we move further away from the ideal case. Finally, 
as we shall shortly see, the computation of the Quality factor for groupings 
involving more than two line segments becomes greatly simplified.

The Quality factor, independent of the associated grouping, allows us to 
condense a great deal of information. Since this factor has the specific 
meaning above, it may be used for quickly indexing other types of symmetries 
or asymmetries within the grouped set by simply calculating the required 
range in Quality factor. 

 Sets of N overlapping parallel line segments in which all possible pairings 
satisfy the overlapping parallelism criterion may be combined to form a larger 
set which we shall call an overlapping parallel bundle of order N. The 
associated Quality factor is defined by:
\vspace{-2mm}
\begin{equation}
 Quality_{OVP}^{N} = 
\frac { \sum^{N}_{j} \sum^{N}_{i>j} LP_{i} + LP_{i} - 
        \sigma^{\parallel}_{i} - \sigma^{\parallel}_{j} }
{
2.0 \times \sum^{N}_{j} \sum^{N}_{i>j} {L^{\it VL}_{i,j}}
}
\label{equ:qualbundle}
\end{equation}
\vspace{-2mm}
 The denominator is essentially twice the sum of the lengths of the Virtual 
Lines associated with each pair. This form of the definition for bundles 
retains the self consistency property associated with pairs as described 
above. Note that for the Quality factor to be meaningful for arbitrarily large
bundles, the choice of the type of overlapping parallel pairs must be uniform 
ie, all fully overlapping, or all partially overlapping. If we 
store the Virtual Line parameters for the overlapping parallel pairs we may 
compute this new quality factor directly from information already available. 

\subsection{Non-overlapping Parallel and Collinear Pairs}

 Given the frame work described in the last section, the definition of
non-overlapping parallel lines follows naturally. In addition to the criterion
involving the acute angle between the two line segments, non-overlapping 
parallel lines must also satisfy 
\vspace{-2mm}
\begin{equation}
 L^{\it VL}_{1,2} > LP_{1} + LP_{2} + 
          \sigma^{\parallel}_{1} + \sigma^{\parallel}_{2}
\end{equation}
\vspace{-2mm}
and the perpendicular distances from the point {\bf P} in 
Figure 2b to the lines of which {\it L1} and {\it L2} are
segments must be greater than the largest standard deviation in the position 
of the line segments perpendicular to their direction ($\sigma^{\perp}_{i}$). 
This additional criterion allows us to distinguish between non-overlapping 
parallel, and collinear line segment pairs. A collinear pair is essentially a 
modified non-overlapping parallel pair such that the perpendicular distance 
from the point {\bf P} to both line segments is less than or equal to 
$\sigma^{\perp}_{i}$.

 We define a perfect non-overlapping parallel pair as one for which the sum of 
the lengths of the line segments is equal to the length of the Virtual Line,
and the orientations of the line segments are equal. The Quality factor for 
this grouping is given by 
\vspace{-2mm}
\begin{equation}
 Quality_{NOVP} = 
\frac 
{ LP_{1} + LP_{2} - \sigma^{\parallel}_{1} - \sigma^{\parallel}_{2} }
{ L^{\it VL}_{1,2} }
\end{equation}
\vspace{-2mm}
 We may also form bundles of N non-overlapping parallel line segments, using 
the same pairwise criterion as for overlapping parallel lines. In this case 
however equation~\ref{equ:qualbundle} is no longer appropriate since the 
Quality factor for the bundle would inevitably decrease as we increase the 
number of lines forming the bundle. The appropriate form which still retains 
the self consistency, and monotonicity property associated with the Quality 
factor for parallel overlapping bundles is:
\vspace{-2mm}
\begin{equation}
 Quality_{NOVP}^{N} = 
\frac { \sum^{N}_{j} \sum^{N}_{i>j} LP_{i} + LP_{i} - 
        \sigma^{\parallel}_{i} - \sigma^{\parallel}_{j} }
{
\max_{i \neq j} {L^{\it VL}_{i,j}}
}
\end{equation}
\vspace{-2mm}
 The choice of the type of bundle (step like, or staggered) is application 
specific. Bundles of collinear lines and their associated Quality 
factor are also formed in the same way. Pairs and bundles of collinear lines 
may be replaced by their associated Virtual Line and such lines 
may be treated as physical lines for the purpose of performing further 
grouping operations.

\subsection{V and L Junctions}

 A V junction is defined as any pair of line segments which intersect, and
whose intersection point either lies on one of the line segments and is
less than $\sigma^{\parallel}_{i} + \sigma^{\perp}_{j}$  away from the end 
points of the line segment, or does not lie on either of the line segments.  
An additional requirement is that the acute angle between the two lines must 
lie in the range $\theta_{min}$ to $\theta_{max}$. In order to avoid 
ambiguity with parallel or collinear pairs, $\theta_{min}$ is chosen to be the
same as the limiting angle used to filter line pairs forming possible parallel 
or collinear groups. L junctions are a special case of a V junction where 
$\theta_{min}$ for L junctions is greater than  $\theta_{max}$ for the V 
junctions, and $\theta_{max}$ for L junctions is $\frac {\pi}{2}$. 
In order to avoid ambiguity with $\lambda$ junctions we also label 
as V junctions any line segment pairs, satisfying the above criteria for V 
junctions, the distance between whose closest end points is less than the 
larger of $\sigma^{\parallel}_{i}$ or $\sigma^{\perp}_{i}$.

 Now a perfect V junction is defined as one in which the intersection point 
{\bf P}, shown in Figure 3a, lies precisely at the end 
points of the line segments. Note that there are now two Virtual Lines which 
share the end point {\bf P}. The points  P1, and P4, denote the remaining end 
points of the Virtual Lines respectively. We now define the Quality factor as
\vspace{-2mm}
\begin{equation}
 Quality_{V Junction} = 
\frac {L_{1} - \sigma^{\parallel}_{1} - \sigma^{\perp}_{2}}{L^{\it VL1}} 
\times 
\frac {L_{2} - \sigma^{\parallel}_{2} - \sigma^{\perp}_{1}}{L^{\it VL2}}
\end{equation}
\vspace{-2mm}
where $L^{\it VLi} (i=1,2)$ are the lengths of the Virtual Lines VL1 and VL2
displayed in Figure 3a. In this case 
we have chosen to multiply the ratios of the lengths of the lines to the 
lengths of the Virtual Lines since we are trying to penalize pairings in which
either line is far away from the junction point. The Quality factor
nevertheless retains the symmetry property described for parallel line 
segment pairs. The Quality factor for L junctions is defined in precisely the
same manor. 

\begin{figure}[htbp]
\vspace*{-10mm}
\special{psfile=Figures/Junction.ps}
\vspace*{6cm}
\vspace*{-5mm}
\caption{V (3a) and $\lambda$ (3b) Junctions}
\label{fig:junction}
\end{figure}

 Since an infinity of shapes may be created using V junctions, the concept of 
Quality as applied to an arbitrary bundle of N junctions becomes meaningless. 
For closed sets of junctions however, such a definition is possible. The 
formation of the closed sets is much simplified by flagging the end points 
of the lines as the closer to, and the further away from, the junction point, 
respectively. The flagging of the end points simplifies the the search for 
other types of bundles such as triplets of lines sharing a common junction 
point. The Quality factor for closed sets of combinations of V and L 
junctions, denoted as $S_{N}$ where N is the number of lines in the set, is 
defined so as to reflect the amount of missing information in the hypothesized 
closed set. The precise steps in the formation of the closed sets is outside
the scope of this article. However it suffices to say that all relevant
junction points required to close the set (eg. to form a square or hexagon 
etc..) are first computed. This then allows us to compute the circumference of 
the hypothesized perfect closed set, and hence the Quality factor as 
\vspace{-2mm}
\begin{equation}
 Quality_{S_{N}} = 
\frac { \sum^{N}_{j}\sum^{N}_{i>j} L_{i}  - \sigma^{\parallel}_{i} - \sigma^{\perp}_{j}}
{
2.0 \times Circumference
}
\end{equation}
\vspace{-1mm}
 Where the subscripts i, j refer only to those pairs of line segments which
form the sides of the closed set. Note that collinear line segment pairs may be
also be used in the formation of the closed set. 

\subsection{$\lambda$ and T Junctions}
 
 A $\lambda$ junction is one where the intercept point lies on one of the line 
segments, and the line segments do not form a V or L Junction. Choosing 
$\theta_{min}$ and $\theta_{max}$ so that they have the same values 
as those used for V Junctions assures consistency. T junctions are a special 
case of $\lambda$ junctions where $\theta_{min}$ for T junctions is equivalent 
to $\theta_{max}$ for the $\lambda$ junctions, and $\theta_{max}$ for T 
junctions is $\frac {\pi}{2}$. There is only one Virtual Line involved in the 
definition of $\lambda$ and T junctions as shown in 
Figure 3b. A perfect $\lambda$ or T junction is one in
which the intersection point lies precisely at the end point of only one of
the line segments. We define the Quality factor for $\lambda$ junctions as
\vspace{-2mm}
\begin{equation}
 Quality_{\lambda Junction} = 
\frac {L_{1} - \sigma^{\parallel}_{1} - \sigma^{\perp}_{2}}{L_{\it VL1}}
\end{equation}
\vspace{-2mm}
where $L_{1}$ is the length of the line which does not include the junction
point, and $L_{\it VL1}$ is the length of the Virtual Line. The Quality 
factor for T junctions is defined in precisely the same manor as for 
$\lambda$ Junctions. It is more important that we do not mislabel possible V 
or L junctions as 
$\lambda$ or T junctions, than vice versa, since the latter may be easily 
rectified when we form higher level groupings, but the former tends to 
propagate to higher levels. This is the main reason for the inclusion of the 
additional criterion, based on proximity of end points, stated in the last 
section. 

\section{Experimental Results}

 The above groupings have been implemented within the LPEG system. 
The form of the definitions of the groupings allows many of the 
computations to be performed in parallel. In addition each of the groupings 
themselves may be computed independently. The best implementation of LPEG 
would therefore be in a massively parallel environment. The left-most image in 
Figure~\ref{fig:poly}, kindly supplied by Dr. R. 
Horaud at the LIFIA institute in Grenoble, is that of a Widget used for test 
purposes. The next two images are the results of the 
Canny edge detector, and annotated extracted line segments using the Hough 
transform, respectively. The extracted line segments have been allocated an 
identification number which is displayed as the annotation in the 
corresponding image. 
\vspace{-18mm}
\begin{figure}[htbp]
\begin{picture}(60,60)
\vspace*{5cm}
\put(0,-1){\framebox(39,45){}}
\put(-1,-1){
\special{psfile=Figures/Polyhedron.PS hscale=0.45 vscale=0.515}
}
\put(0,-1){\framebox(39,45){}}
\put(41,-1){\framebox(39,45){}}
\put(41,0){
\special{psfile=Figures/Polyhedron.canny.PS  hscale=0.45 vscale=0.515}
}
\put(41,-1){\framebox(39,45){}}
\put(82,-1){\framebox(39,45){}}
\put(82,0){
\special{psfile=Figures/Polyhedron.hough.lines.annotated.PS  hscale=0.45 vscale=0.515}
}
\put(82,-1){\framebox(39,45){}}
\end{picture}
\caption{Image used for testing ``Low-Level'' grouping algorithms}
\label{fig:poly}
\end{figure}

 The results of groupings with associated Quality factor greater than or equal 
to 0.5, and using a limiting angle of 0.06 radians, are displayed in 
Table 1. The length of the line segments extracted using the
Hough transform were constrained to be above 10 pixels. Starting with an 
ASCII list of 39 extracted line segments the  
grouping was performed in 0.66 CPU seconds (excluding I/O) on a Sun SLC. 
Each column in the table contains the identification numbers associated with
line segments forming the grouping type indicated at the top of the table. In order
to show how the Quality factor may be used effectively in filtering unlikely
groupings, we have applied the algorithm to the same set of lines and
limiting angles used in the formation of Table 1, but 
using a limiting Quality factor of 0.3. The results  are presented 
in Table 2 and the corresponding processing time was 
0.57 CPU seconds. 

\section{Summary and Conclusions}

 We have also applied the algorithm to other lists of line segments extracted
from images of various indoor scenes. The computation time, using the same 
limiting angle and, for limiting Quality factors of 0.5, range from 3 to 60
seconds for a lists containing approximately 100 to 320 line segments. 
In conclusion we
believe that the formalism presented is sufficiently flexible that it may be
used in varied applications in which the user wishes to extract a limited set
of groupings from a set of line segments. The most important attribute of the
proposed groupings is the Quality factor. This parameter allows us to handle
the combinatorial explosion of groupings by focusing the attention of the
system on those groupings which are near ideal. In general we would initially
look for high Quality groupings, and use these to initiate new image
understanding tasks. Since we may determine a worst-case time scale for the
formation of the groupings, given the number of line segments in the image, the
resources which need to be allocated to the grouping process, and the limiting
Quality factor, may be easily determined. Current work involves the construction
of the intermediate level of the LPEG system which uses these groupings in the
formation of closed sets. 

\begin{thebibliography}{1}

\vspace*{-2mm}
\bibitem[{\em Crowley et al.,} 1989] {crowley89} 
J.L. Crowley, A. Chehikian, J. Kittler, J. Illingworth, J.O. Eklundh, 
G. Granlund, J. Wiklund, E. Granum and H.I. Christensen, Vision as Process 
Technical Annex ESPRIT-BRA 3038, University of Aalborg,1989.
\vspace*{-2mm}
\bibitem[{\em Faugeras,} 1990] {faugeras90}
O. Faugeras (Ed.), Computer Vision, ECCV 90.
\vspace*{-2mm}
\bibitem[{\em Horaud et al.,} 1990] {horaud90}
R. Horaud, and F. Veillon, Finding geometric and relational structures in an
image, First ECCV, 52, 57-77, 1990.
\vspace*{-2mm}
\bibitem[{\em Lowe} 1987] {lowe87}
D.G. Lowe, Three-dimensional object recognition from single two-dimensional 
images, AI 31, 355-395, 1987.
\vspace*{-2mm}
\bibitem[{\em Mohan et al.,} 1989] {mohan89}
R. Mohan and R. Nevatia, Using perceptual organization to extract 3-D 
structures, T-PAMI 11, 1121-1139, 1989.
\vspace*{-2mm}
\bibitem[{\em Weiss et al.,} 1986] {weiss86}
R. Weiss and M. Boldt, Geometric grouping applied to straight lines, 
CVPR, 489-495, 1986.

\end{thebibliography}

\input Tables/Polyhedron1
\input Tables/Polyhedron2

\end{document}