\chapter{Casos de Uso}

\section{Introducci�n}
En este cap�tulo se describen los distintos casos de uso que se implementaron con el fin de aplicar los algoritmos desarrollados en los cap�tulos anteriores. Se busc� generar distintos casos de uso que funcionaran como muestra de las funcionalidades que son posibles de realizar mediante la resoluci�n de los algoritmos mencionados.
\section{Caso de Uso 01}
\subsection{Comentarios sobre el caso de uso}
\subsection{Detalles constructivos}
\section{Caso de Uso 02}
\subsection{Comentarios sobre el caso de uso}
Este caso de uso b�sicamente busca desplegar un video en una superfice dada del mundo real. Esto puede ser de gran inter�s como complemento de contenido para un cuadro o cualquier obra si se piensa en aplicarlo para museos. Es posible por ejemplo, generar un video que sea reproducido dentro de los marcos del propio cuadro, en un extremo o en una superficie cualquiera que resulte interesante desde el punto de vista art�stico. A continuaci�n se explican brevemente algunos detalles para lograr la implementaci�n de este caso de uso.
\subsection{Detalles constructivos}
Para lograr lo propuesto para este caso de uso se implement� un proyecto que proyecta el video en uno de los cuadrados del marcador. De esta manera, de toda la l�gica de estimaci�n de pose, solamente se hace uso de la detecci�n y filtrado. En particular no se hace uso de los resultados del posit. Teniendo entonces detectados los cuatro puntos en los que se quiere reproducir el video parecer�a que el problema est� resuelto. Sin embargo, xcode no permite posicionar en forma directa una vista de video en cualquier conjunto de cuatro puntos. \\
Si simplemente se quiere reproducir un video, y no se quiere procesar el contenido, lo m�s c�modo para hacerlo es utilizar la clase \textit{MPMoviePlayerController} que hereda de \textit{NSObject}. Una alternativa similar es haciendo uso de la clase \textit{MPMoviePlayerViewController} que hereda de \textit{UIViewController} y tiene como �nica propiedad una del tipo \textit{MPMoviePlayerController}. \\
\textit{MPMoviePlayerController} tiene un atributo \textit{view} del tipo \textit{UIView} que es la vista y es este atributo el que se quiere posicionar en los cuatro puntos detectados por el filtro. Un atributo del tipo \textit{UIView} tiene un atributo \textit{frame} que es del tipo \textit{CGRect}\\
\begin{verbatim}
theMovie.view.frame = CGRectMake(0, 0, 60, 60);
\end{verbatim}
En el c�digo anterior \textit{theMovie} es del tipo \textit{MPMoviePlayerController}. De esta manera, se tiene el inconveniente de que en principio cualquier video parecer�a que solamente puede ser reproducido sobre rect�ngulos y no en cualquier pol�gono de cuatro puntos por ejemplo. Sin embargo algo que s� se puede hacer a las instancias de la clase \textit{UIView} es una transformaci�n afin o incluso, de manera m�s gen�rica, una homograf�a. \\
\subsection{\textit{CGAffineTransform} y \textit{CATransform3D}}
La clase \textit{UIView} tiene una propiedad llamada \textit{transform} que es del tipo \textit{CGAffineTransform}. Tambi�n tiene una pripiedad llamada layer\\
\subsection{Resoluci�n de Homograf�a}
A continuaci�n se hace el desarrollo de la resoluci�n del sistema de ecuaciones que se tuvo que resolver para hallar los par�metros de la homograf�a que transforma una imagen de referencia en la imagen que se tiene en cada momento como resultado de la captura de la c�mara. Se asume entonces que se conocen los puntos de referencia y los puntos de referencia transformados (los detectados luego del filtrado de segmentos) y lo que se quiere averiguar es la matriz $h$ que logra dicha transformaci�n. Esta homograf�a 2D-2D se puede expresar en forma matricial, en coordenadas homog�neas de la siguiente manera:
\[
\left( \begin{array}{ccc}
h_{11} & h_{12} & h_{13} \\ 
h_{21} & h_{22} & h_{23} \\
h_{31} & h_{32} & h_{33} 
\end{array} \right)
\left( \begin{array}{c}
x \\ 
y \\
z
\end{array} \right)
=
\left( \begin{array}{c}
i \\
j \\
k
\end{array} \right)
\]
donde la matriz $h$  representa la transformaci�n, el vector $(x,y,z)^t$  representa los puntos de referencia a ser transformados y el vector $(i,j,k)^t$  respresenta los puntos detectados cuadro a cuadro como las esquinas del marcador.\\
Asumiendo un valor unitario para las coordenadas $z$ y $k$ la resoluci�n del sistema se simplifica mucho y no se pierde generalidad. Imponiendo esto entonces, el sistema anterior se puede expresar de la siguiente forma:
\begin{equation}\label{eq_1}
xh_{11} + yh_{12} + h_{13} = i
\end{equation}
\begin{equation}\label{eq_2}
xh_{21} + yh_{22} + h_{23} = j
\end{equation}
\begin{equation}\label{eq_3}
xh_{31} + yh_{32} + h_{33} = 1
\end{equation}
Multiplicando la ecuaci�n \ref{eq_3} por $i$ e igual�ndola a la ecuaci�n \ref{eq_1} se obtiene lo siguiente:
\begin{equation}
xh_{11} + yh_{12} + h_{13} = ixh_{31} + iyh_{32} + ih_{33}
\end{equation}
o lo que es lo mismo:
\begin{equation}\label{eq_4}
xh_{11} + yh_{12} + h_{13} - ixh_{31} - iyh_{32} - ih_{33}=0
\end{equation}
Procediendo de manera an�loga y multiplicando la ecuaci�n \eqref{eq_3} por $j$ e igual�ndola a la ecuaci�n \ref{eq_2} se obtiene lo siguiente:
\begin{equation}
xh_{21} + yh_{22} + h_{23} = jxh_{31} + jyh_{32} + jh_{33}
\end{equation}
o lo que es lo mismo:
\begin{equation}\label{eq_4}
xh_{21} + yh_{22} + h_{23} - jxh_{31} - jyh_{32} - jh_{33}=0
\end{equation}
Las ecuaciones \ref{eq_3} y \ref{eq_4} se pueden expresar en forma matricial, de la siguiente manera:
\[
\left( \begin{array}{ccccccccc}
x & y & 1 & 0 & 0 & 0 & -ix & -iy & -i \\ 
0 & 0 & 0 & x & y & 1 & -jx & -jy & -j
\end{array} \right)
\left( \begin{array}{ccccccccc}
h_{11} \\ 
h_{12} \\
h_{13} \\
h_{21} \\
h_{22} \\
h_{23} \\
h_{31} \\
h_{32} \\
h_{33}
\end{array} \right)
=
\left( \begin{array}{ccccccccc}
0 \\ 
0 \\
0 \\
0 \\
0 \\
0 \\
0 \\
0 \\
0
\end{array} \right)
\]


\section{Caso de Uso 03}
\subsection{Comentarios sobre el caso de uso}
\subsection{Detalles constructivos}
\section{Caso de Uso 04}
\subsection{Comentarios sobre el caso de uso}
\subsection{Detalles constructivos}
