\chapter{Casos de Uso}

\section{Introducci�n}
En este cap�tulo se describen los distintos casos de uso que se implementaron con el fin de aplicar los algoritmos desarrollados en los cap�tulos anteriores. Se busc� generar distintos casos de uso que funcionaran como muestra de las funcionalidades que son posibles de realizar mediante la resoluci�n de los algoritmos mencionados.
\section{Caso de Uso 01}
\subsection{Comentarios sobre el caso de uso}
\subsection{Detalles constructivos}
\section{Caso de Uso 02}
\subsection{Comentarios sobre el caso de uso}
Este caso de uso b�sicamente busca desplegar un video en una superfice dada del mundo real. Esto puede ser de gran inter�s como complemento de contenido para un cuadro o cualquier obra si se piensa en aplicarlo para museos. Es posible por ejemplo, generar un video que sea reproducido dentro de los marcos del propio cuadro, en un extremo o en una superficie cualquiera que resulte interesante desde el punto de vista art�stico. A continuaci�n se explican brevemente algunos detalles para lograr la implementaci�n de este caso de uso.
\subsection{Detalles constructivos}
Para lograr lo propuesto para este caso de uso se implement� un proyecto que proyecta el video en uno de los cuadrados del marcador. De esta manera, de toda la l�gica de estimaci�n de pose, solamente se hace uso de la detecci�n y filtrado. En particular no se hace uso de los resultados del posit. Teniendo entonces detectados los cuatro puntos en los que se quiere reproducir el video parecer�a que el problema est� resuelto. Sin embargo, xcode no permite posicionar en forma directa una vista de video en cualquier conjunto de cuatro puntos. \\
Si simplemente se quiere reproducir un video, y no se quiere procesar el contenido, lo m�s c�modo para hacerlo es utilizar la clase \textit{MPMoviePlayerController} que hereda de \textit{NSObject}. Una alternativa similar es haciendo uso de la clase \textit{MPMoviePlayerViewController} que hereda de \textit{UIViewController} y tiene como �nica propiedad una del tipo \textit{MPMoviePlayerController}. \\
\textit{MPMoviePlayerController} tiene un atributo \textit{view} del tipo \textit{UIView} que es la vista y es este atributo el que se quiere posicionar en los cuatro puntos detectados por el filtro. Un atributo del tipo \textit{UIView} tiene un atributo \textit{frame} que es del tipo \textit{CGRect}\\
\begin{verbatim}
theMovie.view.frame = CGRectMake(0, 0, 60, 60);
\end{verbatim}
En el c�digo anterior \textit{theMovie} es del tipo \textit{MPMoviePlayerController}. De esta manera, se tiene el inconveniente de que en principio cualquier video parecer�a que solamente puede ser reproducido sobre rect�ngulos y no en cualquier pol�gono de cuatro puntos por ejemplo. Sin embargo algo que s� se puede hacer a las instancias de la clase \textit{UIView} es una transformaci�n afin o incluso, de manera m�s gen�rica, una homograf�a. \\
\subsection{\textit{CGAffineTransform} y \textit{CATransform3D}}
La clase \textit{UIView} tiene una propiedad llamada \textit{transform} que del tipo \textit{CGAffineTransform}. Tambi�n tiene una pripiedad llamada \\
\subsection{Resoluci�n de Homograf�a}
C�mo se resolvi� la homograf�a...desarrollo de las cuentas y resoluci�n del sistema de ecuaciones.
\section{Caso de Uso 03}
\subsection{Comentarios sobre el caso de uso}
\subsection{Detalles constructivos}
\section{Caso de Uso 04}
\subsection{Comentarios sobre el caso de uso}
\subsection{Detalles constructivos}