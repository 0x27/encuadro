\chapter{Implementaci�n}

\section{Introducci�n}
En este cap�tulo se muestra la integraci�n de los conocimientos adquiridos para poder llevar a cabo la realidad aumentada en una aplicaci�n real. Si bien era de gran inter�s del proyecto la exploraci�n de distintos m�todos y algoritmos parec�a importante poder poner en pr�ctica todo lo desarrollado en un producto final que pudiera parecerse a un prototipo de aplicaci�n comercial.\\
La aplicaci�n consta de distintas funcionalidades que se describen en este cap�tulo, tales como:\\
 \begin{itemize}
\item[(1)] QR
\item[(2)] Navegaci�n por listas de cuadros
\item[(3)] Servidor
\item[(4)] Detecci�n SIFT
\item[(5)] Diferentes realidades aumentadas seg�n la obra.
\end{itemize}
\section{Diagrama global de la aplicaci�n}
bla bla laaa
\section{TableViewController}
jjjjj\\
\section{QR}
\subsection{QR. Una realidad}
El uso de los identificadores QR (Quick Response), es cada vez m�s generalizado. �ltimamente debido al incremento significativo del uso de \textit{smart devices} el hecho de poder contar con c�mara y poder de procesamiento hace que sea frecuente encontrar aplicaciones con el poder de reconocimiento de QRs. Son usados principalmente para vincular con p�ginas web o en algunos casos tambi�n como tarjetas personales. 
\subsection{Qu� son realmente los QRs?}
Se puede decir que los QRs tienen una funci�n parecido a la de los c�digos de barras pero en forma bidimensional. El hecho de contar con una dimensi�n m�s hace que se cuente con mucho m�s informaci�n que en un c�digo de barras.
Existen distintos tipos de QRs, con distintas capacidades de almacenamiento.
\subsection{Codificaci�n y decodificaci�n de QRs}
Librer�as

\section{Servidor}
sssssssssssssss
\section{SIFT}

\section{Incorporaci�n de la realidad aumentada a la aplicaci�n}
asdfasdfasdf\\

