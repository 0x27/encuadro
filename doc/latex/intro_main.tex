<<<<<<< HEAD
{\documentclass[12pt,a4paper, spanish]{report}
=======
\documentclass[12pt,a4paper, spanish]{report}
>>>>>>> 6fbebac25ef8ea16c16c9b8e5388e5b90ef63888
\usepackage[spanish]{babel}
\usepackage[latin1]{inputenc}  % Ambos para solucin de asuntos de idioma
\usepackage[T1]{fontenc}
\usepackage{tocbibind}  % Bibliografa en el indice
\usepackage{titlesec}  % Posibilidad de editar los formatos de chapter y section
%\usepackage{times}  % Fuente de letras
\usepackage{amsmath,amssymb,mathrsfs,mathptmx}  % Matemticas varias
\usepackage{hyperref} % Para escribir URLs

<<<<<<< HEAD
\usepackage[]{algorithm2e}
\usepackage{listings}
% \usepackage{keyval,fancyvrb,xcolor,float,ifthen,calc,ifplatform}
% \usepackage{minted}

=======
>>>>>>> 6fbebac25ef8ea16c16c9b8e5388e5b90ef63888


% --- Arreglos varios para la inclusion de imgenes
%\usepackage[pdftex]{graphicx}
%\usepackage[dvips]{graphicx}
\usepackage{graphicx}
\usepackage{epstopdf}
<<<<<<< HEAD
% \usepackage{epsfig}
=======
>>>>>>> 6fbebac25ef8ea16c16c9b8e5388e5b90ef63888
\usepackage{float}
\usepackage{subfigure}
%\usepackage{subfig}
\usepackage{wrapfig}
\usepackage[usenames,dvipsnames]{color}
\DeclareGraphicsExtensions{.png,.jpg,.pdf,.mps,.gif,.bmp, .eps}


<<<<<<< HEAD
=======
	

>>>>>>> 6fbebac25ef8ea16c16c9b8e5388e5b90ef63888
\usepackage{multirow}
\usepackage{multicol}
\usepackage{tabulary}
\usepackage[table]{xcolor}
\usepackage{color}
\usepackage{listings}
%\usepackage{subfloat}
\usepackage{tikz}

\setcounter{secnumdepth}{3}
\setcounter{tocdepth}{3}


% --- Para las dimensiones de los mrgenes etc
\frenchspacing \addtolength{\hoffset}{-1.5cm}
\addtolength{\textwidth}{3cm} \addtolength{\voffset}{-2.5cm}
\addtolength{\textheight}{4cm}
% --- Para el encabezado
\usepackage{fancyhdr}
\fancyhead[R]{2012}\fancyhead[L]{enCuadro} \fancyfoot[C]{\thepage}
\pagestyle{fancy}

% --- Formato de la etiqueta Chapter
%\newcommand{\bigrule}{\titlerule[0.5mm]}
%\titleformat{\chapter}[display]{\bfseries\Huge}
%{\Large\chaptertitlename\ \Large\thechapter}
%{0mm} {\filleft} [\vspace{0.5mm} \bigrule]

\titleformat{\chapter}[display]
{\normalfont\Large\filcenter}
{\titlerule[1pt]%
\vspace{1pt}%
\titlerule
\vspace{1pc}%
\LARGE\MakeUppercase{\chaptertitlename} \thechapter}
{1pc}
{\titlerule
\vspace{1pc}%
\Huge}

%-------------------------

\begin{document}
% Esto es para que se muestren todas las referencias aunque no se citen:
\nocite{*}

\renewcommand{\tablename}{Tabla}
\renewcommand{\theenumi}{\Roman{enumi}}
\renewcommand{\labelenumi}{[\textbf{\theenumi}]}
\renewcommand{\thefootnote}{\arabic{footnote}}
% --- Modificacin de entornos enumerate
\renewcommand{\theenumi}{\roman{enumi}}
\renewcommand{\labelenumi}{\theenumi)}
% --- Modificacin de entornos enumerate

% --- Para hacer highlights
\newcommand{\highlAmarillo}[1]{\colorbox{yellow}{#1}}
\newcommand{\highlVerde}[1]{\colorbox{green}{#1}}
\newcommand{\highlRojo}[1]{\colorbox{red}{#1}}

<<<<<<< HEAD

\chapter{Introducci�n}

Debido a la creciente disponibilidad de las plataformas m�viles y el gran poder de procesamiento con el que cuentan, el n�mero de aplicaciones m�viles ha crecido de manera significativa. Dichas plataformas cuentan con sistemas de adquisici�n de audio, video y una variedad de sensores como por ejemplo aceler�metro y giroscopio, lo que las transforma en sistemas ideales para desarrollar aplicaciones de procesamiento multimedia.\\

Por otro lado, desde hace algunos a\~nos varios museos de distintas partes del mundo han comenzado a considerar este tipo de dispositivos, y otras tantas tecnolog�as, como una alternativa muy interesante para brindar un valor agregado al usuario. Proyecciones de im�genes y videos, recorridos interactivos y aplicaciones de \textit{realidad aumentada} son tan s\'olo algunos de los ejemplos. Sin embargo, esta es un �rea muy reciente y en la que todav�a queda un camino muy largo por recorrer.\\

\begin{figure}[h!]
\centering
\includegraphics[scale=0.08]{figs_intro/arIntro.png}
\caption{Ejemplo de realidad aumentada.}
\label{fig: arIntro}
\end{figure}

El presente proyecto busca desarrollar sobre ciertos dispositivos m�viles en particular,  un recorrido interactivo para un museo, con realidad aumentada. Se espera de esta manera, contribuir al desarrollo de herramientas que fomenten contenidos educativos y art�sticos, generando as� un marco para poner la tecnolog�a al servicio de la cultura y la sociedad. As� entonces, se estableci� contacto con dos museos de Montevideo, el ``Museo Nacional de Artes Visuales'' (MNAV) y el ``Museo de Arte Precolombino e Ind�gena'' (MAPI). Se espera basar el prototipo final de la aplicaci�n en obras, piezas arqueol�gicas o mapas informativos, pertenecientes a estos dos museos.\\

Probablemente, la realidad aumentada sea el mayor atractivo del proyecto por ser un �rea que se encuentra en pleno desarrollo y que todo el tiempo recibe ideas innovadoras y muy interesantes, lo que la hace por dem�s apasionante. Vale la pena entonces dar una definici�n para la misma:\\

\textit{La realidad aumentada (AR del ingl�s \textit{Augmented Reality}) es un t�rmino que denota la visi�n de un entorno f�sico del mundo real, cuyos elementos se combinan con elementos virtuales generados por computadora, para la creaci�n de una realidad mixta en tiempo real.}\\

Cuando se genera una imagen por medio de realidad aumentada, conviven en ella elementos reales con elementos virtuales. Es b�sicamente un juego de percepciones. En la Figura \ref{fig: arIntro} se puede ver un ejemplo de ralidad aumentada, que logra plasmar varios de los conceptos anteriores.\\

A lo largo de la presente documentaci�n se espera dar al lector un panorama de lo que fue el proyecto en su conjunto, esperando que sirva como iniciativa para futuros trabajos vinculados a esta rama de la ingenier�a.

\bibliographystyle{unsrt}   
\bibliography{encuadro}  
\end{document}
=======
%



\chapter{Introducci�n}

Debido a la creciente disponibilidad de las plataformas m�viles y el gran poder de procesamiento con el que cuentan, el n�mero de aplicaciones m�viles ha crecido de manera significativa. Dichas plataformas cuentan con sistemas de adquisici�n de audio, video y una variedad de sensores como por ejemplo aceler�metro y giroscopio, lo que las transforma en sistemas ideales para desarrollar aplicaciones de procesamiento multimedia.\\

Por otro lado, desde hace algunos a\~nos varios museos de distintas partes del mundo han comenzado a considerar este tipo de dispositivos, y otras tantas tecnolog�as, como una alternativa muy interesante para brindar un valor agregado al usuario. Proyecciones de im�genes y videos, recorridos interactivos y aplicaciones de \textit{realidad aumentada} son tan s\'olo algunos de los ejemplos. Sin embargo, esta es un �rea muy reciente y en la que todav�a queda un camino muy largo por recorrer.\\

\begin{figure}[h!]
\centering
\includegraphics[scale=0.08]{figs_intro/arIntro.png}
\caption{Ejemplo de realidad aumentada.}
\label{fig: arIntro}
\end{figure}

El presente proyecto busca desarrollar sobre ciertos dispositivos m�viles en particular,  un recorrido interactivo para un museo, con realidad aumentada. Se espera de esta manera, contribuir al desarrollo de herramientas que fomenten contenidos educativos y art�sticos, generando as� un marco para poner la tecnolog�a al servicio de la cultura y la sociedad. As� entonces, se estableci� contacto con dos museos de Montevideo, el ``Museo Nacional de Artes Visuales'' (MNAV) y el ``Museo de Arte Precolombino e Ind�gena'' (MAPI). Se espera basar el prototipo final de la aplicaci�n en obras, piezas arqueol�gicas o mapas informativos, pertenecientes a estos dos museos.\\

Probablemente, la realidad aumentada sea el mayor atractivo del proyecto por ser un �rea que se encuentra en pleno desarrollo y que todo el tiempo recibe ideas innovadoras y muy interesantes, lo que la hace por dem�s apasionante. Vale la pena entonces dar una definici�n para la misma:\\

\textit{La realidad aumentada (AR del ingl�s \textit{Augmented Reality}) es un t�rmino que denota la visi�n de un entorno f�sico del mundo real, cuyos elementos se combinan con elementos virtuales generados por computadora, para la creaci�n de una realidad mixta en tiempo real.}\\

Cuando se genera una imagen por medio de realidad aumentada, conviven en ella elementos reales con elementos virtuales. Es b�sicamente un juego de percepciones. En la Figura \ref{fig: arIntro} se puede ver un ejemplo de ralidad aumentada, que logra plasmar varios de los conceptos anteriores.\\

A lo largo de la presente documentaci�n se espera dar al lector un panorama de lo que fue el proyecto en su conjunto, esperando que sirva como iniciativa para futuros trabajos vinculados a esta rama de la ingenier�a.

% Ejemplo de como hacer una cita:
%\cite{Daniel03simultaneouspose}.



\bibliographystyle{unsrt}   
\bibliography{encuadro}  
\end{document}
>>>>>>> 6fbebac25ef8ea16c16c9b8e5388e5b90ef63888
