\chapter{Hardware y Software}

\section{Introducci�n}
Introducci�n
\section{Software de procesamiento de im�genes}
Software de procesamiento de im�genes
\section{Dispositivos m�viles}
Al trabajar con Apple se cuenta con la ventaja de contar con pocas variantes en cuanto al Hardware utilizado. B�sicamente existen tres tipos de dispositivos en los que se pueden desarrollar: iPhone, iPad y iPod Touch. Para cada variante de plataforma existen distintos modelos que hacen que algunas caracter�sticas importantes como capacidad de procesamiento, resoluci�n de c�mara o tama?o de la pantalla entre otras puedan verse afectadas. A continuaci�n se relata el surgimiento de cada uno de lo dispositivos al mercado y se describen brevemente las principales caracter�sticas.
\subsection{iPhone}
Sin dudas el iPhone fue uno de los saltos m�s grandes en el mundo tecnol�gico en los �ltimos a?os. Logr� llenar el hueco que los PDAs de la d�cada de los 90 no hab�an sabido completar y comenz� a desplazar al invento que revolucion� el mercado de los contenidos de m�sica, el iPod. Gracias a su pantalla t�ctil capacitiva de alta sensibilidad logr� reunir todas las funcionalidades agregando solamente un gran bot�n y algunos extra para controlar volumen o desbloquar el dispositivo. \\
La primera generaci�n del iPhone fue lanzada por Apple en Junio de 2007 en Estados Unidos, luego de una gran inversi�n de la operadora ATT que exig�a exclusividad de venta dentro de dicho pa�s durante los siguientes cuatro a?os. La misma soportaba tecnolog�a GSM cuatribanda y se lanz� en dos variantes de 4GB y 8GB de ROM. El segundo modelo lanz� como novedad el soporte de tecnolog�a 3G cuatribanda y GPS asistido. Luego le siguieron el iPhone 3GS, 4, 4S y el 5, siendo este �ltimo, la sexta y �ltima generaci�n disponible al momento de la redacci�n de este trabajo.\\
Las dimensiones del iPhone 5 son de 58.6 x 123.8 x 7.6 millimetres, resoluci�n de pantalla de 640 x 1136, tiene una velocidad de reloj en la CPU de 1200MHz, RAM de 1GB y la ROM var�a seg�n la variante (16GB, 30GB o 64GB).
\subsection{iPad}
Esta l�nea de dispositivos es la m�s potente en lo que respecta a capacidad de procesamiento.
\subsection{iPod Touch}

 

\section{Entorno de desarrollo}
Entorno de desarrollo
\subsection{xCode y Objective-C}
\subsection{Simulador}
\subsection{Instruments}
\subsubsection{Time Profiler}
\subsubsection{Memory Leak}
\subsection{Librer�as}
\subsubsection{AvFoundation}
\subsubsection{MediaPlayer}
\subsubsection{CoreMotion}
\subsubsection{Tweeter y mensajer�a}
\section{Herramientas}
Herramientas

