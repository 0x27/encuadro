\chapter{LSD: ``Line Segment Detection''}

\section{Introducci�n}
LSD es un algoritmo de detecci�n de segmentos publicado por Rafael Grompone von Gioi, J�r�mie Jakubowicz, Jean-Michel Morel y Gregory Randall en abril de 2010. Es temporalmente lineal, con presici�n inferior a un p�xel y que no requiere de un tuneo previo de par�metros, como casi todos los dem�s algoritmos de id�ntica funci�n; puede ser considerado el estado del arte en cuanto a detecci�n de segmentos en im�genes digitales. Como cualquier otro algoritmo de detecci�n de segmentos, LSD basa su estudio en la b�squeda de contornos angostos dentro de la imagen. Estos son regiones en donde el nivel de brillo de la imagen cambia notoriamente entre p�xeles vecinos, por lo que el gradiente de la misma resulta de vital importancia. Puede verse como una composici�n de tres pasos:\\
\begin{itemize}
\item[(1)] Divisi�n de la imagen en las llamadas \textit{line-support regions} que son grupos conexos de p�xeles con id�ntico gradiente, hasta cierta tolerancia. 
\item[(2)] B�squeda del segmento que mejor aproxime cada  \textit{line-support region}.
\item[(3)] Validaci�n o no de cada segmento detectado en el punto anterior. 
Los puntos (1) y (2) est�n basados en el algoritmo de detecci�n de segmentos de Burns, Hanson y Riseman y el punto (3) es una adaptaci�n del m�todo \textit{a contrario} de Desolneux, Moisan y Morel. 
\end{itemize}

\section{\textit{Line-support regions}}
El primer paso de LSD es el dividir la imagen en regiones conexas de p�xeles con igual gradiente, a menos de cierta tolerancia, llamadas \textit{line-support regions}. El m�todo para realizar tal divisi�n es del tipo ``regi�n creciente''; cada regi�n comienza por un p�xel y cierto �ngulo asociado al gradiente del mismo. Luego, se testean los ocho vecinos de dicho p�xel y los que cuenten con un �ngulo similar al del p�xel original, son inclu�dos en la regi�n. En cada iteraci�n el �ngulo asociado a la regi�n es calculado como el promedio de los �ngulos del gradiente de cada p�xel; la iteraci�n termina cuando ya no se pueden agregar m�s p�xeles a una regi�n.