\chapter{Costos}
\label{chap: costos}

\section{Introducci\'on}
En este cap\'itulo se realiza un an�lisis de los tiempos y costos econ\'omicos del proyecto \textbf{enCuadro}. Por ser un proyecto de investigaci\'on m\'as desarrollo de \textit{software}, el grueso de estos costos se deben a las horas-hombre dedicadas. Se exponen entonces las mismas y su costo estimado, para luego ser comparadas con las horas-hombre estimadas al inicio del proyecto. Finalmente, se comentan los dem�s costos que se tuvieron, que fueron principalmente adquisici�n de tecnolog�a.

\section{Tiempos}

A lo largo de los 15 meses que dur� el proyecto (desde el $1/\text{ago}/2011$ hasta $30/\text{nov}/2012$, descontando enero de $2012$ que no se trabaj�), se anotaron cada una de las horas dedicadas al proyecto por cada uno de los integrantes. Si se asume que todos los integrantes trabajaron en promedio las mismas horas y que a lo largo de todos los meses, la dedicaci�n se mantuvo constante, se puede razonar como sigue:\\
$$ \begin{array}{|cc|} \hline
\text{Horas totales aproximadas dedicadas por }\textbf{enCuadro}: & \textbf{3500\: hs.} \\ \hline
\end{array} $$

Por lo que cada integrante dedic\'o al proyecto unas:

$$ \begin{array}{|cc|} \hline
\text{Horas totales aproximadas dedicadas por cada integrante :} & \textbf{875\:hs.} \\ \hline
\end{array} $$

A lo largo de los 15 meses que dur� el proyecto, se tiene que cada integrante le dedic� aproximadamente unas 58 horas mensuales. A modo de comentario, se menciona que la dedicaci�n total estimada al comienzo del proyecto fue de $4400$ horas a una intensidad aproximada de $25$ horas semanales por integrante, durante un periodo de 11 meses. 

\section{Costos de investigaci\'on m\'as desarrollo}

Si se cotiza cada hora-hombre de forma lineal a unos 15 d�lares, se tiene que el costo en recursos humanos sum�:

$$ \begin{array}{|cc|} \hline
\text{Costo total aproximado en recursos humanos :} & \textbf{USD\: 52.500} \\ \hline
\end{array} $$
\section{Otros costos}

Cualquier proyecto, mayor o menor a \textbf{enCuadro}, tiene costos extra como por ejemplo trasporte, llamadas telef�nicas, consultor�as. Sin embargo, ninguno de estos costos fueron considerables durante el proyecto, y por lo tanto no ser�n tomados en cuenta.\\
 
Sin embargo, como se desarroll� para iOS, hubo que hacerlo en computadoras \textit{Apple} con procesadores \textit{Intel} y sistemas operativos \textit{Mac OS X}. Se decidi� entonces, presentarse a los fondos concursables del ``Programa de Apoyo a la Investigaci�n Estudiantil'' (PAIE) que dependen de la ``Comisi�n Sectorial de Investigaci�n Cient�fica'' (CSIC) con el objetivo de conseguir fondos para comprar los equipos necesarios. Estos fueron obtenidos por un monto en pesos Uruguayos de \textbf{$\$25.000$}.\\
 
De esta manera, se compraron una computadora \textit{Mac mini Mid 2011} y un \textit{iPod Touch} $4^{ta}$ \textit{generaci�n}. En la Tabla \ref{tab: otros_costos} se especifican los costos de cada uno de ellos y el costo total dedicado a la adquisici�n de tecnolog�a que es la suma de ambos.

\begin{table}[h!]
\centering
$$
\begin{array}{|c|c|} \hline
\textbf{Equipo adquirido} & \textbf{costo (USD)} \\ \hline
\textit{Man mini} & 940 \\ \hline
\textit{iPod Touch} & 370 \\ \hline \hline
\textbf{Total} & 1310 \\ \hline
\end{array}
$$
\caption{Costo total dedicado a la adquisici�n de equipamiento para el desarrollo del proyecto.}
\label{tab: otros_costos}
\end{table}

\section{Costos totales}
Una buena aproximaci\'on a lo que fueron los costos totales del proyecto es la suma de los costos dedicados a los recursos humanos destinados a la investigaci\'on m\'as desarrollo, sumados a los costos de la adquisici\'on de tecnolog\'ia necesaria para llevar a cabo el proyecto. Se puede decir entonces, que los costos de \textbf{enCuadro} fueron del orden de:\\

$$ \begin{array}{|cc|} \hline
\text{Costo total del proyecto \textbf{enCuadro} :} & \textbf{USD\: 53.810} \\ \hline
\end{array} $$

\section{Resumen}
A lo largo de este cap\'itulo se vieron los distintos costos econ\'omicos que tuvo este proyecto, y se vio el alto costo que tienen la investigaci\'on y el desarrollo, por requerir de mano de obra calificada a lo largo per\'iodos de tiempo considerables. Se vio cual fue la dedicaci�n real y la estimada al inicio del proyecto y se concluye que se sobrestim� la carga horaria, aunque no se logr� la intensidad de trabajo que se esperaba. Se destaca la importancia que tienen para nuestro pa\'is, los fondos destinados a la investigaci\'on y el desarrollo.\\