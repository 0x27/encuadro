\chapter{Costos}
\section{Introducci\'on}
En este cap\'itulo se realiza un an�lisis de los tiempos y costos del proyecto \textbf{enCuadro}. Por ser un proyecto de investigaci\'on m\'as desarrollo de \textit{software}, el grueso de los costos se deben a las horas-hombre dedicadas. Se exponen entonces las mismas y su costo estimado, para luego ser comparadas con las horas-hombre estimadas al inicio del proyecto. Finalmente, se comentan los dem�s costos que se tuvieron, que fueron principalmente, adquisici�n de tecnolog�a.

\section{Tiempos}

A lo largo de los 15 meses que dur� el proyecto (desde el $1/\text{ago}/2011$ hasta $30/\text{nov}/2012$, descontando enero de $2012$ que no se treabaj�), se anotaron cada una de las horas dedicadas al proyecto por cada uno de los integrantes. Si se asume que todos los integrantes trabajaron en promedio las mismas horas y que a lo largo de todos los meses, la dedicaci�n se mantuvo constante (esta �ltima afirmaci�n es discutible), se puede razonar como sigue:\\
$$ \begin{array}{|cc|} \hline
\text{Horas totales aproximadas dedicadas por }\textbf{enCuadro}: & \textbf{3500} \\ \hline
\end{array} $$

Por lo que cada integrante dedic\'o al proyecto unas:

$$ \begin{array}{|cc|} \hline
\text{Horas totales aproximadas dedicadas por cada integrante :} & \textbf{875} \\ \hline
\end{array} $$

En resumen, a lo largo de los 15 meses que dur� el proyecto, se tiene que cada integrante le dedic� aproximadamente unas 58 horas mensuales.
\section{Costos de investigaci\'on m\'as desarrollo}

Si se cotiza cada hora-hombre de forma lineal, a unos 15 d�lares, se tiene que el costo en recursos humanos sum�:

$$ \begin{array}{|cc|} \hline
\text{Costo total aproximado en recursos humanos :} & \textbf{USD\: 52.200} \\ \hline
\end{array} $$
\section{Otros costos}

Cualquier proyecto, mayor o menor a \textbf{enCuadro}, tiene costos extra como por ejemplo trasporte, llamadas telef�nicas, consultor�as. Sin embargo, ninguno de estos costos fueron considerables durante el proyecto, y por lo tanto no ser�n tomados en cuenta.\\
 
Sin embargo, como se desarroll� para iOS, hubo que hacerlo �nicamente en computadoras \textit{Apple} con procesadores \textit{Intel} y sistemas operativos \textit{Mac OS X}. Se decidi� entonces, presentarse a los fondos concursables del ``Programa de Apoyo a la Investigaci�n Estudiantil'' (PAIE) que dependen de la ``Comisi�n Sectorial de Investigaci�n Cient�fica'' (CSIC) con el objetivo de conseguir fondos para comprar los equipos necesarios. Felizmente, estos fueron obtenidos, por un monto en pesos Uruguayos de \textbf{$\$25.000$}.\\
 
De esta menera, se compraron una computadora \textit{Mac mini} y un \textit{iPod Touch}. En la tabla \ref{tab: otros_costos} se especifican los costos.

\begin{table}
\centering
\begin{array}{|c|c|} \hline

\end{array}

\caption{Hola}
\label{tab: otros_costos}
\end{table}