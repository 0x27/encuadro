\chapter{Realidad aumentada}

Definicion de realidad aumentada, ejemplos. Breve mencion de las cosas que se usan, modelo camara pinhole, calibracion, extraccion de caracteristicas(LSD,ORT), marcador utilizado.
Algoritmos de estimacion de pose. Algoritmos de correspondencias. Toda la rama Posit
En este capitulo se presentan las tecnicas realizadas para implementar la realidad aumentada

\section{Definici�n}
\label{sec:Definici�nAR}
Es posible definir la realidad aumentada (AR del ingl�s \textit{Augmented Reality}) como una vista directa o indirecta en tiempo real, de alg�n elemento o escena f�sica del mundo real, a la que se le agrega informaci�n de manera virtual o digital mediante el uso de herramientas computacionales \cite{furht11}. Cuando se genera una escena por medio de la realidad aumentada, conviven en ella elementos reales con elementos virtuales que buscan verse tan reales como se pueda. Es basicamente un juego de percepciones.\\
La realidad aumentada es un �rea que se encuentra en pleno desarrollo y todo el tiempo aparecen ideas novedosas y muy interesantes, lo que la hace por dem�s apasionante.

\section{Fundamento}
\label{sec:FundamentoAR}
El proceso mediante el cual se logra la realidad aumenada puede dividirse b�sicamente en dos grandes partes. La primera es el seguimiento de la c�mara. Esto es el lograr ubicar la c�mara perfectamente (rotaci�n y traslaci�n), respecto de un eje de coordenadas arbitrario asignado al mundo real. Este seguimiento puede lograrse mediante el reconocimiento de caracter�sticas de las im�genes tomadas por la c�mara o incluso mediante la detecci�n de bordes o esquinas de alg�n marcador en particular correctamente modelado respecto del \textit{eje del mundo} previamente mencionado. Tener un correcto modelo de los objetos del mundo respecto de este eje resulta una cuesti�n realmente importante para poder luego recostru�r la imagen con la infrmaci�n agregada digitalmente. \\
Una vez conocida la posici�n del la c�mara en el mundo, resta agregar la informaci�n a la imagen. Esto es posible ya que al tener un correcto modelo del mundo respecto de su eje de coordenadas, es posible ubicar (en el modelo te�rico) un objeto digital y resolviendo un sencillo sistema de ecuaciones se obtiene su posici�n en la imagen cuadro a cuadro.\\
El proceso mediante el cual se genera una imagen 2D a partir de un modelo 3D se denomina Renderizaci�n y se analizar� con algo m�s de detalle en secciones subsiguientes. Tambi�n se analizar�n m�s a fondo el concepto del modelado 3D y el de modelado de la c�mara en cuesti�n, ya que no todas las c�maras se comportan de igual manera desde que su construcci�n no es id�ntica a la construcci�n de ninguna otra c�mara (aunque bajo ciertas hip�tesis, se deber� suponer que cierto conjunto de c�maras cuenta con caracter�sticas similares). Para realizar entonces la transformaci�n desde el mundo 2D al mundo 3D y viceversa, resulta de vital importancia contar con un modelo detallado y preciso de la c�mara utilizada.\\
\section{Modelado 3D}
\label{sec:Modelo3D}
?Aca que puede ir?
\section{Rendering}
\label{sec:Rendering}
Esta seccion es importante!
